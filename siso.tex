\documentclass[11pt]{article} % For LaTeX2e
\usepackage[utf8]{inputenc}
\usepackage{amsmath, amsfonts, amsthm, amssymb, algorithm, graphicx}
\usepackage{mathtools}
\usepackage{enumitem}
\usepackage{pdfpages}
\usepackage[noend]{algpseudocode}
\usepackage{fancyhdr, lastpage}
\usepackage[vmargin=1.20in,hmargin=1.25in,centering,letterpaper]{geometry}
\setlength{\headsep}{.10in}
\setlength{\headheight}{15pt}
 
% Macros
\DeclareMathOperator{\BigOm}{\mathcal{O}}
\newcommand{\BigOh}[1]{\BigOm\left({#1}\right)}
\DeclareMathOperator{\BigTm}{\Theta}
\newcommand{\BigTheta}[1]{\BigTm\left({#1}\right)}
\DeclareMathOperator{\BigWm}{\Omega}
\newcommand{\BigOmega}[1]{\BigWm\left({#1}\right)}
\DeclareMathOperator{\LittleOm}{\mathrm{o}}
\newcommand{\LittleOh}[1]{\LittleOm\left({#1}\right)}
\DeclareMathOperator{\LittleWm}{\omega}
\newcommand{\LittleOmega}[1]{\LittleWm\left({#1}\right)}

\newcommand{\calY}{\mathcal{Y}}
\newcommand{\calS}{\mathcal{S}}
\newcommand{\calD}{\mathcal{D}}

\newcommand{\calU}{\mathcal{U}}
\newcommand{\Alt}{\mathrm{Alt}}
\newcommand{\kl}{\mathrm{kl}}
\newcommand{\calA}{\mathcal{A}}
\newcommand{\calP}{\mathcal{P}}



\newcommand{\calH}{\mathcal{H}}

\newcommand{\ArgTop}{\mathrm{ArgTop}}
\newcommand{\Top}{\mathrm{Top}}
\newcommand{\TopEps}{\mathrm{Top}^{\epsilon}}
\newcommand{\ArgTopEps}{\mathrm{ArgTop}^{\epsilon}}


\newcommand{\KL}{\mathrm{KL}}
\newcommand{\Sym}{\mathrm{Sym}}
\newcommand{\Alg}{\mathrm{Alg}}
\newcommand{\Sim}{\mathrm{Sim}}


\newcommand{\istar}{i^*}
\newcommand{\im}{\mathrm{im }}
\newcommand{\Z}{\mathbb{Z}}
\newcommand{\C}{\mathbb{C}}
\newcommand{\R}{\mathbb{R}}
\newcommand{\I}{\mathbb{I}}
\newcommand{\Exp}{\mathbb{E}}
\newcommand{\Q}{\mathbb{Q}}
\newcommand{\sign}{\mathrm{sign\ }}
\newcommand{\abs}{\mathrm{abs\ }}
\newcommand{\eps}{\varepsilon}
\newcommand{\Var}{\mathrm{Var}}
\newcommand{\Cov}{\mathrm{Cov}}
\newcommand{\zo}{\{0, 1\}}
\newcommand{\SAT}{\mathit{SAT}}
\renewcommand{\P}{\mathbf{P}}
\newcommand{\NP}{\mathbf{NP}}
\newcommand{\coNP}{\co{NP}}
\newcommand{\co}[1]{\mathbf{co#1}}
\renewcommand{\Pr}{\mathbb{P}}
{
 \theoremstyle{plain}
      \newtheorem{asm}{Assumption}
}
\theoremstyle{plain}
\newtheorem{thm}{Theorem}[section]
\newtheorem{claim}[thm]{Claim}
\newtheorem{lem}[thm]{Lemma}
\newtheorem{cor}[thm]{Corollary}


\newtheorem{prop}[thm]{Proposition}

\theoremstyle{definition}
\newtheorem{defn}{Definition}[section]
\newtheorem{conj}{Conjecture}[section]
\newtheorem{exmp}{Example}[section]
\newtheorem{exc}{Exercise}[section]


\theoremstyle{remark}
\newtheorem*{rem}{Remark}
\newtheorem*{note}{Note}
\usepackage{mathtools}
\DeclarePairedDelimiter\ceil{\lceil}{\rceil}
\DeclarePairedDelimiter\floor{\lfloor}{\rfloor}

\makeatletter
\makeatother
\renewcommand{\theequation}{\arabic{section}.\arabic{equation}}
 
\algnewcommand\algorithmicinput{\textbf{INPUT:}}
\algnewcommand\INPUT{\item[\algorithmicinput]}
\algnewcommand\algorithmicoutput{\textbf{OUTPUT:}}
\algnewcommand\OUTPUT{\item[\algorithmicoutput]}
 
% Formatting Macros
 
\pagestyle{fancy}
%\chead{\sc Problem \hmwkAssignmentNum.\hmwkProblemNum}
%\chead{}
\rhead{\em Max Simchowitz}
\cfoot{}
\lfoot{}
\rfoot{\sc Page\ \thepage\ of\ \protect\pageref{LastPage}}
\renewcommand\headrulewidth{0.4pt}
\renewcommand\footrulewidth{0.4pt}
 
%%% magic code starts
%\mathcode`*=\string"8000
%\begingroup
%\catcode`*=\active
%\xdef*{\noexpand\textup{\string*}}
%\endgroup
%%% magic code ends

\title{Homework 1}

\author{Max Simchowitz\\
msimchow@berkeley.edu}


\begin{document}
\section{PID control}
Our goal is to track a constant signal $r(t) = r_0$, using a PID controller, in the SISO setting. Assume our system has impulse-response function $g(t)$, so that starting from an zero initial state, $y(t) = u(t)*g(t) + d(t)$, where $d(t)$ is some disturbance. 

\section{Point-Estimating $g$}
How to choose $u$ depends on our goal. Suppose we just want to learn the coefficients of $g$. Let $H(u)$ denote the hankel matrix associated with $U$, so that $y(t) = H(u)g + d(t)$. Hence, for one roll out, the optimal robust experiment design to measure $g$ is 
\begin{eqnarray}
\max_{u \in \calU} \sup_{d \in \calD}\|H^{\dagger}(u)y - g\|_2 = \max_{u \in \calU} \sup_{d \in \calD}\|H^{\dagger}(u)d(t)\|_2
\end{eqnarray}
Other objectives would consider different norms. For example, we could try to pick $u$ to minimize the $\|\|_{H_2}$ or $\|\cdot\|_{\infty}$ norm of $g$.
\section{PID - Control usinig state feedback}

We want to lear a controler $F$ so that the system to minimize $\|y - r\|$, using input $y - r$. Thus, the new input to the system is $y(t+1) = (A+BF)y - BFr + BFd(t)$, and our doal is to minimize $\|C(y - r)\|_2^2$. Let's assume that even though our measurements have disturbances, but the true system has no disturbance (we don't need noise rejection). Then, our goal is to choose $u$ and estimate $\hat{B},\hat{A}$ so as to minimize
\begin{eqnarray}
\end{eqnarray}









In an order-one system $g$ is of the form $g(t) = ca^t$. Rather than estimate $g$ directly, we want to learn a controller $k = f(y -r)$ such that to minimize
\begin{eqnarray}
\|y-r\|
\end{eqnarray}
This corresponds to the system $y(t+1) = y$









\end{document}	